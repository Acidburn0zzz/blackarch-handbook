%%%%%%%%%%%%%%%%%%%%%%%%%%%%%%%%%%%%%%%%%%%%%%%%%%%%%%%%%%%%%%%%%%%%%%%%%%%%%%%%
%                                                                              %
% BlackArch Linux Guide                                                        %
%                                                                              %
%%%%%%%%%%%%%%%%%%%%%%%%%%%%%%%%%%%%%%%%%%%%%%%%%%%%%%%%%%%%%%%%%%%%%%%%%%%%%%%%


%%% BEGIN %%%
\documentclass[a4paper, oneside, 11pt]{book}


%%% INCLUDES %%%
\renewcommand{\familydefault}{\sfdefault}
\usepackage{verbatim}
\usepackage[utf8]{inputenc}
\usepackage{geometry}
\usepackage{array}
\usepackage{graphicx}
\usepackage{supertabular}
\usepackage{verbatim}
\usepackage{pstricks}
\usepackage{fancyhdr}
\usepackage{fancyvrb}
\usepackage{tocloft}
\usepackage{color}
\usepackage{enumerate}
\usepackage[french]{babel}
\usepackage[
pdftitle={BlackArch Linux, The BlackArch Linux Guide},
pdfsubject={BlackArch Linux, The BlackArch Linux Guide},
pdfauthor={BlackArch Linux, BlackArch Linux},
pdfkeywords={BlackArch Linux, Penetration Testing, Security, Hacking, Linux},
pdfcenterwindow=true
]{hyperref}


%%% LAYOUT %%%
\setlength{\parindent}{0em}
\setlength{\parskip}{1.5ex plus0.5ex minus0.5ex}
\geometry{left=2.5cm, textwidth=16cm, top=3cm, textheight=25cm, bottom=3cm}
\widowpenalty=2000
\clubpenalty=1000
\frenchspacing
\sloppy
\pagecolor{black}
\color{white}
\hypersetup{pdftex=true, colorlinks=true, breaklinks=true, linkcolor=red,
menucolor=red, pagecolor=red, urlcolor=red}
\setcounter{tocdepth}{10}
\setcounter{secnumdepth}{10}


%%% HEADER / FOOTER %%%
\setlength{\headheight}{1.0cm}
\setlength{\headsep}{1.0cm}
\lhead{{\includegraphics[width=1cm,height=1cm]{logo.png}}}
\rhead{The BlackArch Linux Guide}
%\lfoot{\small{foo}}
%\rfoot{\small{bar}}


%%% OWN MACROS %%%
% put own macros here, like renewcommand newcommand etc.


%%% DOCUMENT %%%
\begin{document}
\pagestyle{empty}
\begin{center}
\begin{figure}[htbp]
\centering
\vspace{1cm}
\Huge{\textbf{BlackArch \color{red}Linux}}\\
\vspace{2cm}
\includegraphics[width=8cm]{logo.png}
\label{fig:logo}
\end{figure}
\vspace{1cm}
\Huge{\textbf{Guide de BlackArch}}\\
\vspace{1cm}
\Large{\color{red}http://www.blackarch.org/}\\
\vspace{0.5cm}
\Large{\today}
\end{center}
\newpage
\tableofcontents
\newpage
\pagestyle{fancy}

%%%%%%%%%%%%%%%%%%%%%%%%%%%%%%%%%%%%%%%%%%%%%%%%%%%%%%%%%%%%%%%%%%%%%%%%%%%%%%%%

\chapter{Introduction}

\section{Qu'est-ce que BlackArch Linux?}
\href{http://www.blackarch.org}{BlackArch Linux} est une légère expansion d'Arch 
Linux destinée aux professionnels de la sécurité informatique.
\\\\
L'ensemble d'outils est distribué tel un  
\href{https://wiki.archlinux.org/index.php/Unofficial\_User\_Repositories}
{dépôt non-officiel} d'Arch Linux, vous permettant d'installer les composantes de
BlackArch sur une installation d'Arch Linux existante. Les paquets peuvent y être
installés individuellement ou par catégorie.
\\\\
Le dépot logiciel de Black Arch contient plus de 650 outils et ce nombre augmente
sans cesse. Tous ces outils sont minutieusement testés avant d'être ajouté aux
dépots, afin de garantir leurs bon fonctionnement.

%\section{The story of BlackArch Linux}
%foo bar

%\section{Supported platforms}
%foo bar

\section{S'impliquer}
Il est possible d'entrer en contact avec l'équipe Black Arch à l'aide de ces
informations:
\\\\
Web: \url{http://www.blackarch.org/}
\\\\
Courriel: \href{mailto:blackarchlinux@gmail.com}{blackarchlinux@gmail.com}
\\\\
IRC: \url{irc://irc.freenode.net/blackarch}

%%%%%%%%%%%%%%%%%%%%%%%%%%%%%%%%%%%%%%%%%%%%%%%%%%%%%%%%%%%%%%%%%%%%%%%%%%%%%%%%

\chapter{Guide d'utilisation}

\section{Installation}
La section suivante démontre comment configurer le dépôt de paquets de Black 
Arch ainsi que la procédure à suivre afin d'installer des logiciels à partir de 
celle-ci. BlackArch supporte l'installation de fichiers binaires précompilés 
ainsi que l'installation à l'aide du code source original des paquets.
\\\\
BlackArch est compatible avec une installation de base d'Arch Linux. Il s'agira 
d'un dépôt non-officiel de paquets. Si vous désirez une image ISO à la place,
visitez la section \href{http://www.blackarch.org/download.html#iso}{Live ISO}.
\\\\

\subsection{Installer le dépôt}
Il suffit de 6 étapes afin d'installer le dépot de BlackArch sur une installation
existante de Arch Linux. Vous devez suivre la recette dans l'ordre.
You must follow the instuctions in order. N'ajoutez pas \textbf{blackarch} 
à votre fichier \textit{pacman.conf} sans avoir complété les étapes 0 à 2.
\begin{enumerate} \itemsep4pt
\item Si vous avez déjà installé BlackArch et qu'il y existe déjà une entrée
\textbf{[blackarch]} dans \textit{/etc/pacman.conf}, vous devrez la supprimer ou
la mettre en commentaire, et ensuite exécuter la commande \textit{pacman -Syy}.
\item Exécutez les commandes suivantes en tant que root. Ces commande
téléchargeront et installeront les clés nécessaire pour garantir l'authenticité
des paquets via une signature digitale.
{\small
\color{gray}
\begin{verbatim}
wget -q http://blackarch.org/keyring/blackarch-keyring.pkg.tar.xz{,.sig}
gpg --keyserver hkp://pgp.mit.edu --recv 4345771566D76038C7FEB43863EC0ADBEA87E4E3
gpg --keyserver-o no-auto-key-retrieve --with-f blackarch-keyring.pkg.tar.xz.sig
pacman-key --init
rm blackarch-keyring.pkg.tar.xz.sig
pacman --noc -U blackarch-keyring.pkg.tar.xz
\end{verbatim}
}
\item Si désiré, vérifiez l'authenticité des clés autant de sources que vous 
désirez afin de vous convaincre de leurs validités.
\item Ajoutez les lignes suivantes à votre fichier \textit{/etc/pacman.conf}:
{\small
\color{gray}
\begin{verbatim}
[blackarch]
Server = <site_miroir>/$repo/os/$arch
\end{verbatim}
}
Replacez \textit{\textless site\_mirroir\textgreater} par le miroir officiel
de votre choix. La liste des miroirs est disponible sur
\href{http://www.blackarch.org/}{notre site web}.

\item Exécutez:
{\small
\color{gray}
\begin{verbatim}
# pacman -Syyu
\end{verbatim}
}
\end{enumerate}

\subsection{Installation de paquets}
You may now install tools from the blackarch repository.
\begin{enumerate}
\item To list all of the available tools, run
{\small
\color{gray}
\begin{verbatim}
# pacman -Sgg | grep blackarch | cut -d' ' -f2 | sort -u
\end{verbatim}
}
\item To install all of the tools, run
{\small
\color{gray}
\begin{verbatim}
# pacman -S blackarch
\end{verbatim}
}
\item To install a category of tools, run
{\small
\color{gray}
\begin{verbatim}
# pacman -S blackarch-<category>
\end{verbatim}
}
\item To see the blackarch categories, run
{\small
\color{gray}
\begin{verbatim}
# pacman -Sg | grep blackarch
\end{verbatim}
}
\end{enumerate}

\subsection{Installing packages from source}
As part of an alternative method of installation, you can build the BlackArch
packages from source. You can find the PKGBUILDs on
\href{https://github.com/BlackArch/blackarch/tree/master/packages}{github}. To
build the entire repo, you can use the
\href{https://github.com/BlackArch/blackman}{blackman} tool.
\begin{itemize}
\item First, you have to install blackman. If the BlackArch package repository
is setup on your machine, you can install blackman:
{\small
\begin{verbatim}
pacman -S blackman
\end{verbatim}
}
\item You can build and install blackman from source:
{\small
\color{gray}
\begin{verbatim}
mkdir blackman
cd blackman
wget https://raw2.github.com/BlackArch/blackarch/master/packages/blackman/PKGBUILD
# Make sure the PKGBUILD has not been maliciously tampered with.
makepkg -s
\end{verbatim}
}
\item Download, compile and install packages:
{\small
\color{gray}
\begin{verbatim}
$ sudo blackman -i package
\end{verbatim}
}
\item Download, compile and install whole category:
{\small
\color{gray}
\begin{verbatim}
$ sudo blackman -g group
\end{verbatim}
}
\item Download, compile and install all of the BlackArch tools:
{\small
\color{gray}
\begin{verbatim}
$ sudo blackman -a
\end{verbatim}
}
\item To list the blackarch categories:
{\small
\color{gray}
\begin{verbatim}
$ blackman -l
\end{verbatim}
}
\item To list category tools:
{\small
\color{gray}
\begin{verbatim}
$ blackman -p category
\end{verbatim}
}
\end{itemize}

\subsection{Installing from live-, netinstall- ISO or ArchLinux}
You can install BlackArch Linux from one of our live- or netinstall-ISOs.\\See
\url{http://www.blackarch.org/download.html#iso}. The following steps are
required after the ISO boot up.
\begin{itemize}
\item Install blackarch-install-scripts package:
{\small
\color{gray}
\begin{verbatim}
$ sudo pacman -S blackarch-install-scripts
\end{verbatim}
}
\item Run
{\small
\color{gray}
\begin{verbatim}
$ sudo blackarch-install
\end{verbatim}
}
\end{itemize}

%%%%%%%%%%%%%%%%%%%%%%%%%%%%%%%%%%%%%%%%%%%%%%%%%%%%%%%%%%%%%%%%%%%%%%%%%%%%%%%%

\chapter{Developer Guide}

\section{Arch's Build System and Repositories}

PKGBUILD files are build scripts. Each one tells makepkg(1) how to create a package. PKGBUILD files
are written in Bash.

For more information, read (or skim through) the following:
\begin{itemize}
	\item \href{https://wiki.archlinux.org/index.php/Creating_Packages}{Arch Wiki: Creating Packages}
	\item \href{https://wiki.archlinux.org/index.php/Makepkg}{Arch Wiki: makepkg}
	\item \href{https://wiki.archlinux.org/index.php/PKGBUILD}{Arch Wiki: PKGBUILD}
	\item \href{https://wiki.archlinux.org/index.php/Arch_Packaging_Standards}{Arch Wiki: Arch Packaging Standards}
\end{itemize}

\section{Repository structure}

You can find the main BlackArch git repo here:
\href{https://github.com/BlackArch/blackarch}{https://github.com/BlackArch/blackarch}. There are
also several secondary repos here:
\href{https://github.com/BlackArch}{https://github.com/BlackArch}.

Within the main git repo, there are three important directories:

\begin{itemize}
	\item docs - Documentation.
	\item packages - PKGBUILD files.
	\item scripts - Useful little scripts.
\end{itemize}

\subsection{Scripts}

Here is a reference for scripts in the \verb|scripts/| directory:

\begin{itemize}
	\item baaur - Soon, this will upload packages to the AUR.
	\item babuild - Build a package.
	\item bachroot - Manage a chroot for testing.
	\item baclean - Clean old .pkg.tar.xz files from the package repo.
	\item baconflict - Soon this will replace \verb|scripts/conflicts|.
	\item bad-files - Find bad files in built packages.
	\item balock - Obtain or release the package repo lock.
	\item banotify - Notify IRC about package pushes.
	\item barelease - Release packages to the package repo.
	\item baright - Print the BlackArch copyright info.
	\item basign - Sign packages.
	\item basign-key - Sign a key.
	\item blackman - This behaves sort of like pacman but builds from git (not to be confused with nrz's blackman).
	\item check-groups - Check groups.
	\item checkpkgs - Check packages for errors.
	\item conflicts - Check for file conflicts.
	\item dbmod - Modify a package database.
	\item depth-list - Create a list sorted by dependency depth.
	\item deptree - Create a dependency tree, listing only blackarch-provided packages.
	\item get-blackarch-deps - Get a list of blackarch dependencies for a package.
	\item get-official - Get official packages for release.
	\item list-loose-packages - List packages that are not in groups and are not dependencies for other packages.
	\item list-needed - List missing dependencies.
	\item list-removed - List packages that are in the package repo but not in git.
	\item list-tools - List tools.
	\item outdated - Look for packages that are out-dated in the package repo relative to the git repo.
	\item pkgmod - Modify a build package.
	\item pkgrel - Increment pkgrel in a package.
	\item prep - Clean up a PKGBUILD file's style and find errors.
	\item sitesync - Sync between a local copy of the package repo and a remote copy.
	\item size-hunt - Hunt for large packages.
	\item source-backup - Backup package source files.
\end{itemize}

\section{Contributing to repository}
This section shows you how to contribute to the BlackArch Linux project. We
accept pull requests of all sizes, from tiny typo fixes to new packages.\\For
help, suggestions, or questions feel free to contact us.
\\\\
Everyone is welcome to contribute. All contributions are appreciated.

\subsection{Required tutorials}
Please read the following tutorials before contributing:
\begin{itemize}
\item
\href{https://wiki.archlinux.org/index.php/Arch\_Packaging\_Standards)}{Arch
Packaging Standards}
\item \href{https://wiki.archlinux.org/index.php/Creating\_Packages}{Creating
Packages}
\item \href{https://wiki.archlinux.org/index.php/PKGBUILD}{PKGBUILD}
\item \href{https://wiki.archlinux.org/index.php/Makepkg}{Makepkg}
\end{itemize}

\subsection{Steps for contributing}
In order to submit your changes to the BlackArchLinux project, follow these
steps:
\begin{enumerate}
\item Fork the repository from
\url{https://github.com/BlackArchLinux/blackarchlinux}
\item Hack the necessary files, (e.g. PKGBUILD, .patch files, etc).
\item Commit your changes.
\item Push your changes.
\item Ask us to merge in your changes, preferably through a pull request.
\end{enumerate}

\subsection{Example}
The following example demonstrates submitting a new package to the BlackArch
project. We use \href{https://wiki.archlinux.org/index.php/yaourt}{yaourt}
(you can use pacaur as well) to fetch a pre-existing PKGBUILD file for
\textbf{nfsshell} from the \href{https://aur.archlinux.org/}{AUR} and adjust it
according to our needs.

\subsubsection{Fetch PKGBUILD}
Fetch the \textit{PKGBUILD} file using yaourt or pacaur:
{\small
\color{gray}
\begin{verbatim}
user@blackarchlinux $ yaourt -G nfsshell
==> Download nfsshell sources
x LICENSE
x PKGBUILD
x gcc.patch
user@blackarchlinux $ cd nfsshell/
\end{verbatim}
}

\subsubsection{Clean up PKGBUILD}
Clean up the \textit{PKGBUILD} file and save some time:
{\small
\color{gray}
\begin{verbatim}
user@blackarchlinux nfsshell $ ./blarckarch/scripts/prep PKGBUILD
cleaning 'PKGBUILD'...
expanding tabs...
removing vim modeline...
removing id comment...
removing contributor and maintainer comments...
squeezing extra blank lines...
removing '|| return'...
removing leading blank line...
removing $pkgname...
removing trailing whitespace...
\end{verbatim}
}

\subsubsection{Adjust PKGBUILD}
Adjust the \textit{PKGBUILD} file:
{\small
\color{gray}
\begin{verbatim}
user@blackarchlinux nfsshell $ vi PKGBUILD
\end{verbatim}
}

\subsubsection{Build the package}
Build the package:
{\small
\color{gray}
\begin{verbatim}
user@blackarchlinux nfsshell $ makepkg -sf
==> Making package: nfsshell 19980519-1 (Mon Dec  2 17:23:51 CET 2013)
==> Checking runtime dependencies...
==> Checking buildtime dependencies...
==> Retrieving sources...
-> Downloading nfsshell.tar.gz...
% Total    % Received % Xferd  Average Speed   Time    Time     Time
CurrentDload  Upload   Total   Spent    Left  Speed100 29213  100 29213    0
0  48150      0 --:--:-- --:--:-- --:--:-- 48206
-> Found gcc.patch
-> Found LICENSE
...
<lots of build process and compiler output here>
...
==> Leaving fakeroot environment.
==> Finished making: nfsshell 19980519-1 (Mon Dec  2 17:23:53 CET 2013)
\end{verbatim}
}

\subsubsection{Install and test the package}
Install and test the package:
{\small
\color{gray}
\begin{verbatim}
user@blackarchlinux nfsshell $ pacman -U nfsshell-19980519-1-x86_64.pkg.tar.xz
user@blackarchlinux nfsshell $ nfsshell # test it
\end{verbatim}
}

\subsubsection{Add, commit and push package}
Add, commit and push the package
{\small
\color{gray}
\begin{verbatim}
user@blackarchlinux nfsshell $ cd /blackarchlinux/packages
user@blackarchlinux ~/blackarchlinux/packages $ mv ~/nfsshell .
user@blackarchlinux ~/blackarchlinux/packages $ git add nfsshell && git commit
nfsshell && git push
\end{verbatim}
}

\subsubsection{Create a pull request}
Create a pull request on \href{https://github.com/}{github.com}
{\small
\color{gray}
\begin{verbatim}
firefox https://github.com/<contributor>/blackarchlinux
\end{verbatim}
}

\subsection{Requests}
\begin{enumerate}
\item Don't add \textbf{Maintainer} or \textbf{Contributor} comments to
\textit{PKGBUILD} files. Add maintainer and contributor names to the
AUTHORS section of BlackArch guide.
\item For the sake of consistency, please follow the general style of the other
\textit{PKGBUILD} files in the repo and use two-space indentation.
\end{enumerate}

\subsection{General tips}
\href{http://wiki.archlinux.org/index.php/Namcap}{namcap} can check packages for
errors.

%%% APPENDIX %%%
\appendix
\include{appendix}

\end{document}

%%% EOF %%%
