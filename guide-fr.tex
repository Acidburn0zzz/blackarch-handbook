%%%%%%%%%%%%%%%%%%%%%%%%%%%%%%%%%%%%%%%%%%%%%%%%%%%%%%%%%%%%%%%%%%%%%%%%%%%%%%%%
%                                                                              %
% BlackArch Linux Guide                                                        %
%                                                                              %
%%%%%%%%%%%%%%%%%%%%%%%%%%%%%%%%%%%%%%%%%%%%%%%%%%%%%%%%%%%%%%%%%%%%%%%%%%%%%%%%


%%% BEGIN %%%
\documentclass[a4paper, oneside, 11pt]{book}


%%% INCLUDES %%%
\renewcommand{\familydefault}{\sfdefault}
\usepackage{verbatim}
\usepackage[utf8]{inputenc}
\usepackage{geometry}
\usepackage{array}
\usepackage{graphicx}
\usepackage{supertabular}
\usepackage{verbatim}
\usepackage{pstricks}
\usepackage{fancyhdr}
\usepackage{fancyvrb}
\usepackage{tocloft}
\usepackage{color}
\usepackage{enumerate}
\usepackage[french]{babel}
\usepackage[
pdftitle={BlackArch Linux, The BlackArch Linux Guide},
pdfsubject={BlackArch Linux, The BlackArch Linux Guide},
pdfauthor={BlackArch Linux, BlackArch Linux},
pdfkeywords={BlackArch Linux, Penetration Testing, Security, Hacking, Linux},
pdfcenterwindow=true
]{hyperref}


%%% LAYOUT %%%
\setlength{\parindent}{0em}
\setlength{\parskip}{1.5ex plus0.5ex minus0.5ex}
\geometry{left=2.5cm, textwidth=16cm, top=3cm, textheight=25cm, bottom=3cm}
\widowpenalty=2000
\clubpenalty=1000
\frenchspacing
\sloppy
\pagecolor{black}
\color{white}
\hypersetup{pdftex=true, colorlinks=true, breaklinks=true, linkcolor=red,
menucolor=red, pagecolor=red, urlcolor=red}
\setcounter{tocdepth}{10}
\setcounter{secnumdepth}{10}


%%% HEADER / FOOTER %%%
\setlength{\headheight}{1.0cm}
\setlength{\headsep}{1.0cm}
\lhead{{\includegraphics[width=1cm,height=1cm]{logo.png}}}
\rhead{The BlackArch Linux Guide}
%\lfoot{\small{foo}}
%\rfoot{\small{bar}}


%%% OWN MACROS %%%
% put own macros here, like renewcommand newcommand etc.


%%% DOCUMENT %%%
\begin{document}
\pagestyle{empty}
\begin{center}
\begin{figure}[htbp]
\centering
\vspace{1cm}
\Huge{\textbf{BlackArch \color{red}Linux}}\\
\vspace{2cm}
\includegraphics[width=8cm]{logo.png}
\label{fig:logo}
\end{figure}
\vspace{1cm}
\Huge{\textbf{Guide de BlackArch}}\\
\vspace{1cm}
\Large{\color{red}http://www.blackarch.org/}\\
\vspace{0.5cm}
\Large{\today}
\end{center}
\newpage
\tableofcontents
\newpage
\pagestyle{fancy}

%%%%%%%%%%%%%%%%%%%%%%%%%%%%%%%%%%%%%%%%%%%%%%%%%%%%%%%%%%%%%%%%%%%%%%%%%%%%%%%%

\chapter{Introduction}

\section{Qu'est-ce que BlackArch Linux?}
\href{http://www.blackarch.org}{BlackArch Linux} est une légère expansion d'Arch 
Linux destinée aux professionnels de la sécurité informatique.
\\\\
L'ensemble d'outils est distribué tel un  
\href{https://wiki.archlinux.org/index.php/Unofficial\_User\_Repositories}
{dépôt non-officiel} d'Arch Linux, vous permettant d'installer les composantes de
BlackArch sur une installation d'Arch Linux existante. Les paquets peuvent y être
installés individuellement ou par catégorie.
\\\\
Le dépot logiciel de Black Arch contient plus de 650 outils et ce nombre augmente
sans cesse. Tous ces outils sont minutieusement testés avant d'être ajouté aux
dépots, afin de garantir leurs bon fonctionnement.

%\section{The story of BlackArch Linux}
%foo bar

%\section{Supported platforms}
%foo bar

\section{S'impliquer}
Il est possible d'entrer en contact avec l'équipe Black Arch à l'aide de ces
informations:
\\\\
Web: \url{http://www.blackarch.org/}
\\\\
Courriel: \href{mailto:blackarchlinux@gmail.com}{blackarchlinux@gmail.com}
\\\\
IRC: \url{irc://irc.freenode.net/blackarch}

%%%%%%%%%%%%%%%%%%%%%%%%%%%%%%%%%%%%%%%%%%%%%%%%%%%%%%%%%%%%%%%%%%%%%%%%%%%%%%%%

\chapter{Guide d'utilisation}

\section{Installation}
La section suivante démontre comment configurer le dépôt de paquets de Black 
Arch ainsi que la procédure à suivre afin d'installer des logiciels à partir de 
celle-ci. BlackArch supporte l'installation de fichiers binaires précompilés 
ainsi que l'installation à l'aide du code source original des paquets.
\\\\
BlackArch est compatible avec une installation de base d'Arch Linux. Il s'agira 
d'un dépôt non-officiel de paquets. Si vous désirez une image ISO à la place,
visitez la section \href{http://www.blackarch.org/download.html#iso}{Live ISO}.
\\\\

\subsection{Installer le dépôt}
Il suffit de 6 étapes afin d'installer le dépot de BlackArch sur une installation
existante de Arch Linux. Vous devez suivre la recette dans l'ordre.
You must follow the instuctions in order. N'ajoutez pas \textbf{blackarch} 
à votre fichier \textit{pacman.conf} sans avoir complété les étapes 0 à 2.
\begin{enumerate} \itemsep4pt
\item Si vous avez déjà installé BlackArch et qu'il y existe déjà une entrée
\textbf{[blackarch]} dans \textit{/etc/pacman.conf}, vous devrez la supprimer ou
la mettre en commentaire, et ensuite exécuter la commande \textit{pacman -Syy}.
\item Exécutez les commandes suivantes en tant que root. Ces commande
téléchargeront et installeront les clés nécessaire pour garantir l'authenticité
des paquets via une signature digitale.
{\small
\color{gray}
\begin{verbatim}
wget -q http://blackarch.org/keyring/blackarch-keyring.pkg.tar.xz{,.sig}
gpg --keyserver hkp://pgp.mit.edu --recv 4345771566D76038C7FEB43863EC0ADBEA87E4E3
gpg --keyserver-o no-auto-key-retrieve --with-f blackarch-keyring.pkg.tar.xz.sig
pacman-key --init
rm blackarch-keyring.pkg.tar.xz.sig
pacman --noc -U blackarch-keyring.pkg.tar.xz
\end{verbatim}
}
\item Si désiré, vérifiez l'authenticité des clés autant de sources que vous 
désirez afin de vous convaincre de leurs validités.
\item Ajoutez les lignes suivantes à votre fichier \textit{/etc/pacman.conf}:
{\small
\color{gray}
\begin{verbatim}
[blackarch]
Server = <site_miroir>/$repo/os/$arch
\end{verbatim}
}
Replacez \textit{\textless site\_mirroir\textgreater} par le miroir officiel
de votre choix. La liste des miroirs est disponible sur
\href{http://www.blackarch.org/}{notre site web}.

\item Exécutez:
{\small
\color{gray}
\begin{verbatim}
# pacman -Syyu
\end{verbatim}
}
\end{enumerate}

\subsection{Installation de paquets}
Vous pouvez désormais installer des outils directement du dépôt de BlackArch
\begin{enumerate}
\item Afin d'obtenir la liste de tous les outils disponibles exécutez
{\small
\color{gray}
\begin{verbatim}
# pacman -Sgg | grep blackarch | cut -d' ' -f2 | sort -u
\end{verbatim}
}
\item Afin d'installer tous les outils, exécutez
{\small
\color{gray}
\begin{verbatim}
# pacman -S blackarch
\end{verbatim}
}
\item Afin d'installer une catégorie complète d'outils, exécutez
{\small
\color{gray}
\begin{verbatim}
# pacman -S blackarch-<nom de la catégorie>
\end{verbatim}
}
\item Afin d'obtenir la liste des catégories de paquets disponibles, exécutez
{\small
\color{gray}
\begin{verbatim}
# pacman -Sg | grep blackarch
\end{verbatim}
}
\end{enumerate}

\subsection{Installation des paquets depuis le code source}
Il existe une méthode alternative à l'installation de paquets sous forme binaire.
BlackArch vous offre la possibilité d'obtenir le code source directement du 
dépôt officiel du paquet désiré, et BlackArch le compilera pour vous. Vous 
trouverez les fichiers PKGBUILD sur notre 
\href{https://github.com/BlackArch/blackarch/tree/master/packages}{github}. Afin
d'obtenir le dépôt en entier, vous pourrez utiliser l'outil 
\href{https://github.com/BlackArch/blackman}{blackman}.
\begin{itemize} \itemsep10pt
\item Vous devez premièrement installer blackman. Si vous avez déjà configuré
le dépôt de paquets de BlackArch sur votre système, tel qu'indiqué au début de 
ce document, vous vous pouvez installer blackman à l'aide de pacman à l'aide de
cette commande:
{\small
\color{gray}
\begin{verbatim}
pacman -S blackman
\end{verbatim}
}
\item Vous pouvez aussi installer blackman directement du code source:
{\small
\color{gray}
\begin{verbatim}
mkdir blackman
cd blackman
wget https://raw2.github.com/BlackArch/blackarch/master/packages/blackman/PKGBUILD
# Make sure the PKGBUILD has not been maliciously tampered with.
makepkg -s
\end{verbatim}
}
\item Afin de télécharger et compiler un paquet:
{\small
\color{gray}
\begin{verbatim}
$ sudo blackman -i package
\end{verbatim}
}
\item Afin de télécharger, compiler, et installer une catégorie entière:
{\small
\color{gray}
\begin{verbatim}
$ sudo blackman -g group
\end{verbatim}
}
\item Afin de télécharger, compiler, et installer tous les outils présents
dans le dépôt de paquets de BlackArch:
{\small
\color{gray}
\begin{verbatim}
$ sudo blackman -a
\end{verbatim}
}
\item Afin d'obtenir la liste de toutes les catégories de paquets de BlackArch:
{\small
\color{gray}
\begin{verbatim}
$ blackman -l
\end{verbatim}
}
\item Si vous voulez ensuite obtenir la liste de tous les paquets compris dans
une catégorie particulière, vous pouvez obtenir la liste de ceux-ci ainsi qu'une
courte description pour chaque paquets à l'aide de la commande suivante:
{\small
\color{gray}
\begin{verbatim}
$ blackman -p <nom de la catégorie>
\end{verbatim}
}
\end{itemize}

\subsection{Installation d'un LiveCD, netinstall, ou d'Arch Linux}
Vous pouvez installer Black Arch d'un de nos images ISO, que ce soit un LiveCD
ou un netcd. Vous les trouverez sur \href{http://www.blackarch.org/download.html#iso}{notre site}. Les étapes suivantes sont nécéssaire afin de créer une image bootable:
\begin{itemize}
\item Installez le paquet blackarch-install-scripts:
{\small
\color{gray}
\begin{verbatim}
$ sudo pacman -S blackarch-install-scripts
\end{verbatim}
}
\item Ensuite exécutez:
{\small
\color{gray}
\begin{verbatim}
$ sudo blackarch-install
\end{verbatim}
}
\end{itemize}

%%%%%%%%%%%%%%%%%%%%%%%%%%%%%%%%%%%%%%%%%%%%%%%%%%%%%%%%%%%%%%%%%%%%%%%%%%%%%%%%


\chapter{Guide de développement}

\section{Le Arch Build System et les dépôts de paquets}

Les fichiers PKGBUILD sont de simples scripts permettant d'installer et de 
compiler et installer un programme sur plusieurs environnements différents. 
Chaque PKGBUILD indique à la commande makepkg(1) comment créer un paquet valide
qui pourra ensuite être installé. Les fichiers PKGBUILD sont écrits en bash.

Pour plus d'information, il est suggéré de lire ces pages:
\begin{itemize}
\item \href{http://wiki.archlinux.fr/Standard_paquetage}{Arch Wiki FR:
Standard paquetage}
\item \href{http://wiki.archlinux.fr/Makepkg}{Arch Wiki FR: Makepkg}
\item \href{http://wiki.archlinux.fr/PKGBUILD}{Arch Wiki FR: PKGBUILD}
\end{itemize}

\section{Le standard de paquetage de BlackArch}

Afin de garde les choses simples, nos PKGBUILDs sont similaires avec ceux du
Arch User Repository, à quelques différences près. Tous les paquets doivent
être membre du groupe \textit{blackarch}. Il est possible qu'un paquet soit
membre de plus d'un groupe.

\subsection{Groupes}

Afin de permettre aux utilisateurs d'installer certaines catégories de paquets
exclusivement, ceux-ci ont étés répertoriés en plusieurs groupes. Ces groupes
permettent aux utilisateurs d'exécuter la commande 
\textit{pacman -S \textless nom\_du\_groupe \textgreater } afin d'installer tous
 les paquets compris dans ce groupe.

\subsubsection{blackarch}

En théorie, le groupe \textit{blackarch} doit contenir tous les paquets qu'offre
BlackArch. La majorité des paquets devraient être membre de ce groupe.

\subsubsection{blackarch-anti-forensic}

Groupe contenant les paquets utilisés pour contrer les tentatives 
d'investigation informatique de systèmes. Vous y trouverez des paquets offrants
des solutions de cryptage de données, stéganographie, modification d'attributs 
de fichiers, et tout autre type de logiciels permettant la modification d'un
système de fichiers dans le but d'y masquer de l'information.

Exemples: luks, TrueCrypt, Timestomp, dd, ropeadope, secure-delete

\subsubsection{blackarch-automation}

Groupe contenant les paquets utilisés pour l'automation de plusieurs tâches.
Puisque les paquets présents dans ce groupe varient énormément en 
fonctionnalités, il vous est suggéré d'y consulter directement les paquets
afin de vous informer sur leurs capacités.

Exemples: blueranger, tiger, wiffy

\subsubsection{blackarch-backdoor}


Groupe contenant les paquets reliés à l'exploitation ou à l'ouverture de 
backdoors sur de systèmes vulnérables.

Exemples: backdoor-factory, rrs, weevely

\subsubsection{blackarch-binary}

Groupe contenant les paquets reliés aux fichiers binaires. Vous y trouverez des
paquets permettant l'extraction d'archives d'un firmware, des logiciels de
rétro-ingénierie, logiciels aidant à l'exploitation de fichiers binaires, et
autres.

Exemples: binwally, packerid, hex2bin, binwalk

\subsubsection{blackarch-bluetooth}

Groupe contenant les paquets aidant à l'exploitation du standard Bluetooth
(802.15.1).

Exemples: ubertooth, tbear, redfang

\subsubsection{blackarch-code-audit}

Groupe contenant les paquets à fin d'audit de code source. Ces paquets analysent
statiquement le code source de votre projet afin d'y déceler des vulnérabilités.

Exemples: flawfinder, pscan

\subsubsection{blackarch-cracker}

Groupe contenant les paquets utilisés afin de tester la robustesses de multiples
fonctions cryptographiques.

Exemples: hashcat, john, crunch

\subsubsection{blackarch-crypto}

Groupe contenant tous types de paquets ayant rapport de proche ou de loin à la
cryptographie, que ce soit de l'analyse de ou au craquage de hache.

Exemples: ciphertest, xortool, sbd

\subsubsection{blackarch-database}

Groupe contenant les paquets reliés à l'exploitation de bases de données.

Exemples: metacoretex, blindsql

\subsubsection{blackarch-debugger}

Groupe contenant les paquets permettant à l'usager d’investiguer le
fonctionnement interne d'un programme lors de son utilisation.

Exemples: radare2, shellnoob

\subsubsection{blackarch-decompiler}

Groupe contenant les paquets qui tentent d'effectuer de la rétro-ingénierie sur
un programme compilé.

Exemples: flasm, jd-gui

\subsubsection{blackarch-defensive}

Groupe contenant les paquets utilisés afin de protéger un système des logiciels
et usagers malveillants.

Exemples: arpon, chkrootkit, sniffjoke

\subsubsection{blackarch-disassembler}

Ce groupe est fortement similaire à \textit{blackarch-decompiler} dans le sens
que ces deux groupes tentent d'effectuer une rétro-ingénierie générique de
fichiers exécutables, mais les paquets compris dans
\textit{blackarch-disassembler} en extrairont le code assembleur à la place d'un
pseudocode.

Exemples: inguma, radare2

\subsubsection{blackarch-dos}

Groupe contenant les paquets aidant aux attaques de types \textit{déni de service}.

Exemples: 42zip, nkiller2

\subsubsection{blackarch-drone}

Groupe contenant les paquets aidant à l'entretien et l'utilisation de drones.

Exemples: meshdeck, skyjack

\subsubsection{blackarch-exploitation}

Groupe contenant les paquets utilisés lors de l'exploitation de multiples 
programmes et services.

Exemples: armitage, metasploit, zarp

\subsubsection{blackarch-fingerprint}

Groupe contenant les paquets utilisés à fin de reconnaissance réseautique.

Exemples: dns-map, p0f, httprint

\subsubsection{blackarch-firmware}

Groupe contenant les paquets utilisés lors de l'exploitation de firmware.

Exemples: None yet, amend asap.

\subsubsection{blackarch-forensic}

Groupe contenant les paquets utilisés lors d'investigation informatique portant
sur le recouvrement de données potentiellement effacés.

Exemples: aesfix, nfex, wyd

\subsubsection{blackarch-fuzzer}

Groupe contenant les paquets aidant au fuzzing. 

Exemples: msf, mdk3, wfuzz

\subsubsection{blackarch-hardware}

Groupe contenant les paquets aidant à l'exploitation de matériel physique.

Exemples: arduino, smali

\subsubsection{blackarch-honeypot}

Groupe contenant les paquets agissant comme des \textit{honeypots}, c'est-à-dire
des services sécuritaire, mais qui laissent croire à un usager malveillant
autrement afin d'observer son comportement.

Exemples: artillery, bluepot, wifi-honey

\subsubsection{blackarch-keylogger}

Groupe contenant les paquets utilisés lors d'installation de \textit{keylogging}
, soit l'enregistrement de touches clavier sur un système qui pourrait être
utilisé par un usager tiers afin d'en soutirer des informations confidentielles.

Exemples: klogger, logkeys, xspy

\subsubsection{blackarch-malware}

Groupe contenant tous types de paquets ayant un lien au \textit{malware}, que ce
soit à l'utilisation de ceux-ci ou à leur détection.

Exemples: malwaredetect, peepdf, yara

\subsubsection{blackarch-misc}

Groupe contenant les paquets ne semblant pas prendre part dans aucunes autres
catégories.

Exemples: oh-my-zsh-git, winexe, stompy

\subsubsection{blackarch-mobile}

Groupe contenant les paquets aidant aux activités d'entretien et d'audits 
d'applications, de développement d'application, et de matériel mobile..

Exemples: android-sdk-platform-tools, android-udev-rules

\subsubsection{blackarch-networking}

Groupe contenant les paquets aidant à multiples tâches reliés à la réseautique.

Exemples: Anything pretty much

\subsubsection{blackarch-nfc}

Groupe contenant les paquets utilisant la technologie \textit{NFC} (near-field
communications).

Exemples: nfcutils

\subsubsection{blackarch-packer}

Groupe contenant les paquets utilisant de loin ou de près les \textit{packers},
soit des logiciels qui créent des applications contenant un programme 
malveillant caché à l'intérieur.

Exemples: packerid

\subsubsection{blackarch-proxy}

Groupe contenant les paquets agissant tel un proxy, soit un logiciel redirigeant
le trafic internet afin de l'analyser ou de le modifier.

Exemples: burpsuite, ratproxy, sslnuke

\subsubsection{blackarch-recon}

Groupe contenant les paquets servant à la recherche de services exploitable dans
le monde réel.

Exemples: canri, dnsrecon, netmask

\subsubsection{blackarch-reversing}

Groupe contenant les paquets de décompilation, de désassemblage, et tout autres
types de programmes ayant à voir avec la rétro-ingénierie. 

Exemples: capstone, radare2, zerowine

\subsubsection{blackarch-scanner}

Groupe contenant les paquets utilisés afin de scanner multiples systèmes afin
d'y déceler des vulnérabilités actives.

Exemples: scanssh, tiger, zmap

\subsubsection{blackarch-sniffer}

Groupes contenant les paquets aidant dans l'analyse de trafic réseau.

Exemples: hexinject, pytactle, xspy

\subsubsection{blackarch-social}

Groupe contenant les paquets attaquant principalement les sites de réseaux
sociaux.

Exemples: jigsaw, websploit

\subsubsection{blackarch-spoof}

Groupe contenant les paquets qui tentent de dissimuler un attaquant, de tel que
la victime ne puisse le percevoir comme un acteur malicieux.

Exemples: arpoison, lans, netcommander

\subsubsection{blackarch-threat-model}

Groupe contenant les paquets utilisé pour enregistré et créer des rapports sur
plusieurs modèles de menaces pouvant se produire lors d'un scénario spécifique.

Exemples: magictree

\subsubsection{blackarch-tunnel}

Groupe contenant les paquets utilisés afin d'encapsuler le contenu réseau via un
tiers parti, afin de dissimuler la provenance initiale de données. 

Exemples: ctunnel, iodine, ptunnel

\subsubsection{blackarch-unpacker}

Groupe contenant les paquets utilisés lors de l'extraction de logiciel
malveillants de possibles logiciels \textit{packés} ou obfuscé.

Exemples: js-beautify

\subsubsection{blackarch-voip}

Groupe contenant les paquets utilisés afin d'effectuer des actions ou l'analyse
de la téléphonie VoIP.

Exemples: iaxflood, rtp-flood, teardown

\subsubsection{blackarch-webapp}

Groupe contenant les paquets effectuant des opérations de validation de sécurité
sur des applications accessible via l'internet, afin d'en garantir leurs 
robustesse. 

Exemples: metoscan, whatweb, zaproxy

\subsubsection{blackarch-windows}

Groupe contenant les paquets Windows qui s'exécuteront via Wine.

Exemples: 3proxy-win32, pwdump, winexe

\subsubsection{blackarch-wireless}

Groupe contenant les paquets utilisés lors d'audits de sécurité sur une 
infrastructure sans-fil.

Exemples: airpwn, mdk3, wiffy

\section{Structure du dépôt}

Vous pouvez trouver le dépot principal de BlackArch à 
\href{https://github.com/BlackArch/blackarch}
{https://github.com/BlackArch/blackarch}. Il existe aussi plusieurs dépots de
code à \href{https://github.com/BlackArch}{https://github.com/BlackArch}.

À l'intérieur du dépôt principal, vous trouverez les trois dossiers principaux,
soit :

\begin{itemize}
\item docs - Contient la documentation du projet
\item packages - Contient les fichiers PKGBUILD
\item scripts - Contient multiples scripts d'automation
\end{itemize}

\subsection{Scripts}

Voici une référence des scripts que vous trouverez dans le dossier
\verb|scripts/| :

\begin{itemize}
\item baaur - Soon, this will upload packages to the AUR.
\item babuild - Build a package.
\item bachroot - Manage a chroot for testing.
\item baclean - Clean old .pkg.tar.xz files from the package repo.
\item baconflict - Soon this will replace \verb|scripts/conflicts|.
\item bad-files - Find bad files in built packages.
\item balock - Obtain or release the package repo lock.
\item banotify - Notify IRC about package pushes.
\item barelease - Release packages to the package repo.
\item baright - Print the BlackArch copyright info.
\item basign - Sign packages.
\item basign-key - Sign a key.
\item blackman - This behaves sort of like pacman but builds from git (not to be confused with nrz's Blackman).
\item check-groups - Check groups.
\item checkpkgs - Check packages for errors.
\item conflicts - Check for file conflicts.
\item dbmod - Modify a package database.
\item depth-list - Create a list sorted by dependency depth.
\item deptree - Create a dependency tree, listing only blackarch-provided packages.
\item get-blackarch-deps - Get a list of blackarch dependencies for a package.
\item get-official - Get official packages for release.
\item list-loose-packages - List packages that are not in groups and are not dependencies for other packages.
\item list-needed - List missing dependencies.
\item list-removed - List packages that are in the package repo but not in git.
\item list-tools - List tools.
\item outdated - Look for packages that are out-dated in the package repo relative to the git repo.
\item pkgmod - Modify a build package.
\item pkgrel - Increment pkgrel in a package.
\item prep - Clean up a PKGBUILD file's style and find errors.
\item sitesync - Sync between a local copy of the package repo and a remote copy.
\item size-hunt - Hunt for large packages.
\item source-backup - Backup package source files.
\end{itemize}

\section{Contributing to repository}
This section shows you how to contribute to the BlackArch Linux project. We
accept pull requests of all sizes, from tiny typo fixes to new packages.\\For
help, suggestions, or questions feel free to contact us.
\\\\
Everyone is welcome to contribute. All contributions are appreciated.

\subsection{Required tutorials}
Please read the following tutorials before contributing:
\begin{itemize}
\item
\href{https://wiki.archlinux.org/index.php/Arch\_Packaging\_Standards)}{Arch
Packaging Standards}
\item \href{https://wiki.archlinux.org/index.php/Creating\_Packages}{Creating
Packages}
\item \href{https://wiki.archlinux.org/index.php/PKGBUILD}{PKGBUILD}
\item \href{https://wiki.archlinux.org/index.php/Makepkg}{Makepkg}
\end{itemize}

\subsection{Steps for contributing}
In order to submit your changes to the BlackArchLinux project, follow these
steps:
\begin{enumerate}
\item Fork the repository from
\url{https://github.com/BlackArchLinux/blackarchlinux}
\item Hack the necessary files, (e.g. PKGBUILD, .patch files, etc).
\item Commit your changes.
\item Push your changes.
\item Ask us to merge in your changes, preferably through a pull request.
\end{enumerate}

\subsection{Example}
The following example demonstrates submitting a new package to the BlackArch
project. We use \href{https://wiki.archlinux.org/index.php/yaourt}{yaourt}
(you can use pacaur as well) to fetch a pre-existing PKGBUILD file for
\textbf{nfsshell} from the \href{https://aur.archlinux.org/}{AUR} and adjust it
according to our needs.

\subsubsection{Fetch PKGBUILD}
Fetch the \textit{PKGBUILD} file using yaourt or pacaur:
{\small
\color{gray}
\begin{verbatim}
user@blackarchlinux $ yaourt -G nfsshell
==> Download nfsshell sources
x LICENSE
x PKGBUILD
x gcc.patch
user@blackarchlinux $ cd nfsshell/
\end{verbatim}
}

\subsubsection{Clean up PKGBUILD}
Clean up the \textit{PKGBUILD} file and save some time:
{\small
\color{gray}
\begin{verbatim}
user@blackarchlinux nfsshell $ ./blarckarch/scripts/prep PKGBUILD
cleaning 'PKGBUILD'...
expanding tabs...
removing vim modeline...
removing id comment...
removing contributor and maintainer comments...
squeezing extra blank lines...
removing '|| return'...
removing leading blank line...
removing $pkgname...
removing trailing whitespace...
\end{verbatim}
}

\subsubsection{Adjust PKGBUILD}
Adjust the \textit{PKGBUILD} file:
{\small
\color{gray}
\begin{verbatim}
user@blackarchlinux nfsshell $ vi PKGBUILD
\end{verbatim}
}

\subsubsection{Build the package}
Build the package:
{\small
\color{gray}
\begin{verbatim}
user@blackarchlinux nfsshell $ makepkg -sf
==> Making package: nfsshell 19980519-1 (Mon Dec 2 17:23:51 CET 2013)
==> Checking runtime dependencies...
==> Checking buildtime dependencies...
==> Retrieving sources...
-> Downloading nfsshell.tar.gz...
% Total % Received % Xferd Average Speed Time Time Time
CurrentDload Upload Total Spent Left Speed100 29213 100 29213 0
0 48150 0 --:--:-- --:--:-- --:--:-- 48206
-> Found gcc.patch
-> Found LICENSE
...
<lots of build process and compiler output here>
...
==> Leaving fakeroot environment.
==> Finished making: nfsshell 19980519-1 (Mon Dec 2 17:23:53 CET 2013)
\end{verbatim}
}

\subsubsection{Install and test the package}
Install and test the package:
{\small
\color{gray}
\begin{verbatim}
user@blackarchlinux nfsshell $ pacman -U nfsshell-19980519-1-x86_64.pkg.tar.xz
user@blackarchlinux nfsshell $ nfsshell # test it
\end{verbatim}
}

\subsubsection{Add, commit and push package}
Add, commit and push the package
{\small
\color{gray}
\begin{verbatim}
user@blackarchlinux nfsshell $ cd /blackarchlinux/packages
user@blackarchlinux ~/blackarchlinux/packages $ mv ~/nfsshell .
user@blackarchlinux ~/blackarchlinux/packages $ git commit -am nfsshell && git push
\end{verbatim}
}

\subsubsection{Create a pull request}
Create a pull request on \href{https://github.com/}{github.com}
{\small
\color{gray}
\begin{verbatim}
firefox https://github.com/<contributor>/blackarchlinux
\end{verbatim}
}

\subsubsection{Adding a remote for upstream}
A smart thing to do if you're working upstream and on a fork is to pull your own fork and add the main ba repo as a remote
{\small
\color{gray}
\begin{verbatim}
user@blackarchlinux ~/blackarchlinux $ git remote -v
origin <the url of your fork> (fetch)
origin <the url of your fork> (push)
user@blackarchlinux ~/blackarchlinux $ git remote add upstream https://github.com/blackarch/blackarch
user@blackarchlinux ~/blackarchlinux $ git remote -v
origin <the url of your fork> (fetch)
origin <the url of your fork> (push)
upstream https://github.com/blackarch/blackarch (fetch)
upstream https://github.com/blackarch/blackarch (push)
\end{verbatim}
By defualt, git should push straight to origin, but make sure your git config is configured correctly. This won't be an issue unless you have commit rights as you won't be able to push upstream without them.

If you do have the ability to commit, you might have more success using \textit{git@github.com:blackarch/blackarch.git} but it's up to you.
}
\subsection{Requests}
\begin{enumerate}
\item Don't add \textbf{Maintainer} or \textbf{Contributor} comments to
\textit{PKGBUILD} files. Add maintainer and contributor names to the
AUTHORS section of BlackArch guide.
\item For the sake of consistency, please follow the general style of the other
\textit{PKGBUILD} files in the repo and use two-space indentation.
\end{enumerate}

\subsection{General tips}
\href{http://wiki.archlinux.org/index.php/Namcap}{namcap} can check packages for
errors.

%%% APPENDIX %%%
\appendix
\include{appendix}

\end{document}

%%% EOF %%%